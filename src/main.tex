% ! Author = eunsung
%! Date = 2023/09/11
\nocite{*}

% Preamble
\documentclass[11pt]{article}

% Packages
\usepackage[utf8]{inputenc}
\usepackage{import}

\usepackage{kotex}
\usepackage{textcomp}
\usepackage{authoraftertitle}
\usepackage[framemethod=tikz]{mdframed}
\usepackage[shortlabels]{enumitem}
\usepackage{multicol}
\usepackage{etoolbox}

\usepackage{a4wide}
\usepackage{amsmath,amsthm,amssymb}
\usepackage{bbm, dsfont}
\usepackage{hyperref}
\usepackage{graphicx,xcolor}
\usepackage{tikz,ytableau}
\usepackage{bm}

\usetikzlibrary{decorations.pathreplacing}
\ytableausetup{aligntableaux=bottom}
\numberwithin{equation}{section}

% Macros
\newtheorem{thm}{Theorem}[section]
\newtheorem{lem}[thm]{Lemma}
\newtheorem{prop}[thm]{Proposition}
\newtheorem{cor}[thm]{Corollary}
\theoremstyle{definition}
\newtheorem{exam}[thm]{Example}
\newtheorem{defn}[thm]{Definition}
\newtheorem{notation}[thm]{Notation}
\newtheorem{conj}[thm]{Conjecture}
\newtheorem{question}[thm]{Question}
\newtheorem{problem}[thm]{Problem}
\newtheorem{remark}[thm]{Remark}
\newtheorem*{note}{Note}

% Options
% % page settings
\usepackage{geometry}
\geometry{
	paper=a4paper,
	% total={170mm,257mm},
	top=25mm,
	bottom=25mm,
	left=15mm,
	right=15mm,
	marginparsep=0mm,
	marginparwidth=0mm,
	headsep=5mm,
	footskip=10mm,
}

% one/two column
\onecolumn

% twocolumn separate line
\setlength{\columnseprule}{0.4pt}

% 문제 간격 설정
\usepackage{./options/nvverttab}
\nvnumvtabs{2}

% 색 설정
\usepackage{xcolor}

% 줄 간격 설정
\usepackage{setspace}
\setstretch{1.4}

% 들여쓰기 설정
\setlength\parindent{0pt}

% \usepackage{fancyhdr}

\fancypagestyle{pages}{
	% Headers
	\fancyhead[L]{\MyTitle}
	\fancyhead[C]{}
	\fancyhead[R]{}
	\renewcommand{\headrulewidth}{1.4pt}
	% Footers
	\fancyfoot[L]{}
	\fancyfoot[C]{- \thepage\ -}
	\fancyfoot[R]{\MyAuthor}
	\renewcommand{\footrulewidth}{1.4pt}
}

\pagestyle{pages}

% \usepackage{kotex}
\usepackage{graphicx}
\usepackage{textcomp}
\usepackage{authoraftertitle}
\usepackage[framemethod=tikz]{mdframed}
\usepackage{amsmath}
\usepackage{amssymb}
\usepackage{amsthm}
\usepackage[shortlabels]{enumitem}
\usepackage{multicol}
\usepackage{etoolbox}

\DeclareMathOperator{\sech}{sech}
\DeclareMathOperator{\csch}{csch}

% \import{./options/commands}
% \import{./options/numbering}

% \newtheorem{solution}{}[section]


\newcommand{\innerproduct}[2]{\langle #1, #2 \rangle}


% Document
\begin{document}

\setcounter{tocdepth}{5}

\tableofcontents

\clearpage

\section{Boolean functions and the Fourier expansions}

\subsection*{Summary}
\addcontentsline{toc}{subsection}{Summary}

\begin{defn}\label{defn:bool_func}
	Let $n$ be a positive integer. A \textbf{Boolean function} on $n$ variables is a function $f:\{0,1\}^n \to \{0,1\}$ or $f:\{\pm 1\}^n \to \{\pm 1\}$.
\end{defn}

List of some Boolean functions
\begin{itemize}
	\item $\min_n$, $\max_n$ : minimum and maximum function
	\item $Maj_n$ : majority function
	\item
\end{itemize}

For each point $a = \{a_1, \cdots, a_n\} \in \{\pm 1\}^n$, the indicator polynomial
$$1_{\{a\}}(x) = \prod_{i} \left(\frac{1+a_i x_i}{2}\right)$$
takes value 1 when $x=a$ and value 0 otherwise.

\begin{defn}\label{defn:fourier_expansion}
	Let $f:\{\pm 1\}^n \to \{\pm 1\}$ be a Boolean function. The \textbf{Fourier expansion} of $f$ is the following expression:
	$$f(x) = \sum_{S \subseteq [n]} \hat{f}(S) x^S$$
	where $\hat{f}(S)$ are real numbers called the \textbf{Fourier coefficients} of $f$.

	For $S \subseteq [n]$, we define $\chi_S: \mathbb{F}_2^n \rightarrow \mathbb{R}$ by
	$$\chi_S(x) = (-1)^{\langle x, S \rangle}$$
	where $\langle x, S \rangle = \sum_{i \in S} x_i$. We call $\chi_S$ the \textbf{character} of $S$.
	Further, $\chi_S$ satisfies $\chi_S(x+y) = \chi_S(x) \chi_S(y)$.
\end{defn}

\begin{defn}\label{defn:inner_prod}
	Let $f,g: \{\pm 1\}^n \to \{\pm 1\}$ be Boolean functions. The \textbf{inner product} of $f$ and $g$ is defined by
	$$\langle f, g \rangle = \frac{1}{2^n} \sum_{x \in \{\pm 1\}^n} f(x) g(x)$$
\end{defn}

\begin{thm}\label{thm:fourier_expansion}
	Let $f:\{\pm 1\}^n \to \{\pm 1\}$ be a Boolean function. Then
	$$\hat{f}(S) = \langle f, \chi_S \rangle$$
\end{thm}

\begin{prop}\label{fourier_coef}
Let $f:\{\pm 1\}^n \to \{\pm 1\}$ be a Boolean function. Then
	$$\hat{f}(S) = \langle f, \chi_S \rangle = \mathbb{E}_{\bm{x} ~ } \frac{1}{2^n} \sum_{x \in \{\pm 1\}^n} f(x) \chi_S(x)$$
\end{prop}

\clearpage

\subsection*{Solutions}
\addcontentsline{toc}{subsection}{Solutions}

\begin{enumerate}[\thesection.\arabic{enumi}]
	\item
	\begin{enumerate}[(a)]
		\item\label{\thesection.1(a)} $\min_2(x_1, x_2) =
			\begin{cases}
				1 & \text{if } (x_1, x_2) = (1, 1) \\
				-1 & \text{otherwise}
			\end{cases}$
			
			which means $\min_2(x_1, x_2) = -1 + 2 \times \mathds{1}_{\{(1, 1)\}}(x)$
			with $\mathds{1}_{\{(1, 1)\}}(x) = \prod_{i=1}^2 \frac{1+x_i}{2}$. 
			
			Therefore $\min_2(x_1, x_2) = -1 + 2 \times \prod_{i=1}^2 \frac{1+x_i}{2} = -\frac{1}{2} + \frac{1}{2} x_1 + \frac{1}{2} x_2 + \frac{1}{2} x_3$.

		\item Similar to (a), 
		
		$\min_3(x_1, x_2, x_3) = -1 + 2 \times \prod_{i=1}^3 \frac{1+x_i}{2} = -\frac{1}{4} + \frac{1}{4}\sum_i x_i + \frac{1}{4} \sum_{i, j} x_i x_j + x_1 x_2 x_3$

		\item In $\{ 0, 1 \}$ setting, $\chi_S(x) = (-1)^{\sum_{i \in S} x_i}$. The inner product is 
		\begin{align*}
			\innerproduct{\mathds{1}_{\{a\}}}{\chi_S} 
			&= \mathds{E}_{x \sim \{0, 1\}^n} \left[ \mathds{1}_{\{a\}}(x) \chi_S(x) \right] \\
			&= \frac{1}{2^n} \chi_S (x) \\
			&= \frac{1}{2^n} (-1)^{\sum_{i \in S} x_i}
		\end{align*}

		Hence, the Fourier expansion of $\mathds{1}_{\{a\}}$ is
		\begin{align*}
			\mathds{1}_{\{a\}}(x) 
			&= \sum_{S \subseteq [n]} \frac{1}{2^n} \chi_S(a)\chi_S(x) \\
			&= \frac{1}{2^n} \sum_{S \subseteq [n]} \chi_S(a+x) \\
			&= \frac{1}{2^n} \sum_{S \subseteq [n]} (-1)^{\sum_{i \in S} (a_i + x_i)}
		\end{align*}

		\item $\phi_{\{a\}}(x) = \mathds{1}_{\{a\}}(x) = \frac{1}{2^n} \sum_{S \subseteq [n]} (-1)^{\sum_{i \in S} (a_i + x_i)}$
		\item \begin{align*}
			\phi_{\{a, a+e_i\}} 
			&= \frac{1}{2} \left( \mathds{1}_{\{a\}} + \mathds{1}_{\{a+e_i\}} \right) \\
			&= \frac{1}{2} \cdot \frac{1}{2^n} \sum_{S \subseteq [n]} \left( (-1)^{\sum_{j \in S} a_j} + (-1)^{\sum_{j \in S} a_j + \delta_{ij}} \right) \chi_S(x)
		\end{align*}
		The term $(-1)^{\sum_{j \in S} a_j} + (-1)^{\sum_{j \in S} a_j + \delta_{ij}}$ is 0 if $i \in S$ and $2(-1)^{\sum_{j \in S} a_j}$ otherwise. Therefore,
		\begin{align*}
			\phi_{\{a, a+e_i\}} 
			&= \frac{1}{2} \cdot \frac{1}{2^n} \sum_{S \subseteq [n] \setminus \{i\}} (-1)^{\sum_{j \in S} a_j} \chi_S(x) \\
			&= \frac{1}{2^n} \sum_{S \subseteq [n] \setminus \{i\}} (-1)^{\sum_{j \in S} (a_j + x_j)}
		\end{align*}
		\item Let $\phi$ be the probability density function of each $x_i$, respectively. Then $f = \prod_1^n \phi(x_i)$ 
		and the fourier coefficient is 
		\begin{align*}
			\hat{f}(S) 
			&= \innerproduct{f}{\chi_S} \\
			&= \mathds{E}_{x \sim \{\pm 1\}^n} \left[ f(x) \chi_S(x) \right] \\
			&= \mathds{E}_{x \sim \{\pm 1\}^n} \left[ \prod_{i=1}^n \phi(x_i) \chi_S(x) \right] \\
		\end{align*}
	
	\end{enumerate}
	\item ghi


\end{enumerate}


\clearpage

\bibliography{main}
\bibliographystyle{plain}


\end{document}
