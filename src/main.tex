%! Author = eunsung
%! Date = 2023/09/11

% Preamble
\documentclass[11pt]{article}

% Packages
\usepackage[utf8]{inputenc}
\usepackage{import}

\usepackage{a4wide}
\usepackage{amsmath,amsthm,amssymb}
\usepackage{hyperref}
\usepackage{graphicx,xcolor}
\usepackage{tikz,ytableau}
\usepackage{bm}
\usetikzlibrary{decorations.pathreplacing}
\ytableausetup{aligntableaux=bottom}
\numberwithin{equation}{section}

\newtheorem{thm}{Theorem}[section]
\newtheorem{lem}[thm]{Lemma}
\newtheorem{prop}[thm]{Proposition}
\newtheorem{cor}[thm]{Corollary}
\theoremstyle{definition}
\newtheorem{exam}[thm]{Example}
\newtheorem{defn}[thm]{Definition}
\newtheorem{notation}[thm]{Notation}
\newtheorem{conj}[thm]{Conjecture}
\newtheorem{question}[thm]{Question}
\newtheorem{problem}[thm]{Problem}
\newtheorem{remark}[thm]{Remark}
\newtheorem*{note}{Note}

% Options
\import{./options}{page.tex}
\import{./options}{header.tex}
\import{./options}{packages.tex}
% \import{./options}{commands.tex}
% \import{./options}{numbering.tex}

% \newtheorem{solution}{}[section]

\newtheorem{solinner}{}[section]
\newenvironment{sol}[1]{%
  \renewcommand\thesolinner{#1}%
  \solinner
}{\endsolinner}



% Document
\begin{document}

\setcounter{tocdepth}{3}

\tableofcontents

\clearpage

\section{Boolean functions and the Fourier expansions}

\begin{defn}\label{defn:bool_func}
	Let $n$ be a positive integer. A \textbf{Boolean function} on $n$ variables is a function $f:\{0,1\}^n \to \{0,1\}$ or $f:\{\pm 1\}^n \to \{\pm 1\}$.
\end{defn}

List of some Boolean functions
\begin{itemize}
	\item $\min_n$, $\max_n$ : minimum and maximum function
	\item $Maj_n$ : majority function
	\item
\end{itemize}

For each point $a = \{a_1, \cdots, a_n\} \in \{\pm 1\}^n$, the indicator polynomial
$$1_{\{a\}}(x) = \prod_i \left(\frac{1+a_i x_i}{2}\right)$$
tkaes value 1 when $x=a$ and value 0 otherwise.

\begin{defn}\label{defn:fourier_expansion}
	Let $f:\{\pm 1\}^n \to \{\pm 1\}$ be a Boolean function. The \textbf{Fourier expansion} of $f$ is the following expression:
	$$f(x) = \sum_{S \subseteq [n]} \hat{f}(S) x^S$$
	where $\hat{f}(S)$ are real numbers called the \textbf{Fourier coefficients} of $f$.

	For $S \subseteq [n]$, we define $\chi_S: \mathbb{F}_2^n \rightarrow \mathbb{R}$ by
	$$\chi_S(x) = (-1)^{\langle x, S \rangle}$$
	where $\langle x, S \rangle = \sum_{i \in S} x_i$. We call $\chi_S$ the \textbf{character} of $S$.
	Further, $\chi_S$ satisfies $\chi_S(x+y) = \chi_S(x) \chi_S(y)$.
\end{defn}

\begin{defn}\label{defn:inner_prod}
	Let $f,g: \{\pm 1\}^n \to \{\pm 1\}$ be Boolean functions. The \textbf{inner product} of $f$ and $g$ is defined by
	$$\langle f, g \rangle = \frac{1}{2^n} \sum_{x \in \{\pm 1\}^n} f(x) g(x)$$
\end{defn}

\begin{thm}\label{thm:fourier_expansion}
	Let $f:\{\pm 1\}^n \to \{\pm 1\}$ be a Boolean function. Then
	$$\hat{f}(S) = \langle f, \chi_S \rangle$$
\end{thm}

\begin{prop}\label{fourier_coef}
Let $f:\{\pm 1\}^n \to \{\pm 1\}$ be a Boolean function. Then
	$$\hat{f}(S) = \langle f, \chi_S \rangle = \mathbb{E}_{\bm{x} ~ } \frac{1}{2^n} \sum_{x \in \{\pm 1\}^n} f(x) \chi_S(x)$$
\end{prop}

\clearpage

\subsection{Solutions}

\begin{sol}{1.1(a)}\label{1.1(a)}
	$min_2(x) = 1 \quad \iff \quad x = (1, 1)$. 
\end{sol}


\begin{sol}{A}[title]\label{baz}
This is a theorem.
\end{sol}


\clearpage

\bibliography{main}
\bibliographystyle{plain}


\end{document}
